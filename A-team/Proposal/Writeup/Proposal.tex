\documentclass[12pt,a4paper]{article}
\usepackage{natbib}
\usepackage[british]{babel}
\usepackage{parskip}
\usepackage{amsmath}
\usepackage{graphicx}
\usepackage{hyperref}

%Any global settings go here
\graphicspath{{Figures/}}

%some shortcut commands
\newcommand{\bi}{\begin{itemize}}
\newcommand{\ei}{\end{itemize}}
\newcommand{\be}{\begin{enumerate}}
\newcommand{\ee}{\end{enumerate}}

\begin{document}



\begin{titlepage}

\title{Investigating the effectiveness of diversification strategies based on alternative risk measures\\
\line(1,0){250}\\
\large Proposal}

\date{\today}

\author{Richard Montgomery (MNTRIC006)\\
Tayla Radmore (RDMTAY001)\\
Thomas K{\"o}nigkr{\"a}mer (KNGTHO005)}

\maketitle

\begin{figure}

\begin{center}
\includegraphics[width=5cm]{UCTlogo.jpg}\\[1cm]
\end{center}

\end{figure}

\begin{center}
\line(1,0){150}\\
{\tt mntric006@myuct.ac.za}\\
{\tt rdmtay001@myuct.ac.za}\\
{\tt thomasekng@gmail.com}


keywords: Value-at-Risk, Expected Shortfall, Downside deviance, volatility, risk-budgeting, diversification.
\end{center}

\thispagestyle{empty}

\end{titlepage}

\pagenumbering{roman}

\tableofcontents

\newpage

\pagenumbering{arabic}

\citep{bruder2012managing}\\
\cite{dowd2006after}
% Tom why are we randomly citing stuff here

\section{Introduction}
\label{sec:Intro}
Equity portfolios are created by allocating various weightings to the shares comprising the portfolio. These weightings are 


\section{Research Question}
\label{sec:ResQues}


\section{Literature Review}
\label{sec:LitRev}

\subsection{Major Literature}
\label{subsec:MajLit}
To date some quantile-based risk measures have been looked into. Namely "Value-at-Risk (VaR), coherent risk measures, spectral risk measures, and distortion risk measures" \citep{dowd2006after}.

 \cite{dowd2006after} believe that VaR is an imperfect measure. It may well still be reasonable to look into VaR as it is a well known measure and is not particularly complicated for someone in the relevant field of work to understand. Looking further into VaR, \cite{dowd2006after} feel that there are some positive aspects. Namely that VaR can be used to compare portfolios that are not restricted to a certain type of asset; VaR takes into account the portfolios internal correlations in turn allowing a consideration of number of variables changing at once and lastly it's result is easy to understand and has a linked probability.  \cite{dowd2006after} believe a significant limitation of VaR is that it gives no information about what happens if that worst case lower tail end occurs and they feel that this makes this particular measure a poor measure to be used as a risk target measure. \cite{ACERBI20021505} believe VaR is not a good measure because it does not meet the axioms of coherence.

The next group of risk measures looked into were coherent risk measures and in particular Expected Shortfall \citep{ACERBI20021505}. \cite{dowd2006after} feel that this is still a measure that is easy to generate. They do, however, draw attention to the fact that this particular measure suggests the investor is risk-neutral past that lower tail, this may be seen as problematic. Expected Shortfall will ideally be looked into in much greater detain before starting back-testing.

This progressed to looking into spectral risk measures as these measures align with coherence and risk aversion theory \citep{dowd2006after}. Here \cite{dowd2006after} highlight a number of issues with risk aversion theory, which will not be expanded on at this point but it is worthwhile to keep in mind that risk aversion theory is not a perfect or universally accepted idea. Ideally spectral risk measures will be looked into in much greater detail going forward.

Lastly Distortion measures were briefly looked into, in particular the Wang Transform and a generalisation of it  \citep{dowd2006after}. \cite{dowd2006after} believed that a strength of this measure is its ability to recover the Capital Asset Pricing Model as well as Black-Scholes and that it is a superior measure compared to expected shortfall.
 
It is worth mentioning that \cite{dowd2006after} state in their paper that with regards to risk measures it may often be a case of the best measure for a particular application and not a case of there being a certain risk measure that is better than all the others. 

A justification for replacing VaR with one of these other measures provided by \cite{dowd2006after} is that it would require very little extra work. 

\subsection{Further Reading}
\label{subsec:FurRead}

In addition to the works discussed, further ideas should be looked into. Firstly looking at other risk measures that could be used as well as looking into the previously mentioned measures further. Secondly investigating methods to use these risk measures to diversify a portfolio. 

\cite{dowd2006after} mentioned a number of other risk measures, that may be worthwhile to look into briefly. If some generate any interest they can be looked into further and perhaps added to this paper. These measures were:
\bi
\item Convex risk measures
\item Dynamic risk measures
\item Comonotonicity approaches
\item Markov bounds approaches
\item “Best practices” risk measures
\ei
 
 "Best practice" risk measures may be particularly interesting to look into during the literature review stage as they focus on the idea that the context of the problem may be the most important element \citep{dowd2006after}; however; this would be something extremely difficult and time consuming to back-test and may therefore be beyond the scope of this particular paper.

A particular paper that is worth drawing attention to is Conditional Value-at-Risk, spectral risk measures and
(non-)diversification in portfolio selection problems - A comparison with mean–variance analysis by Mario Brandtner. This appears to be a particularly useful paper as it may well be following a similar structure as the paper being proposed

Some further papers that may aid this process are: 
\bi
\item 
\item 
\item
\ei

\section{Methodology \& Data}
\label{sec:Method}
Firstly we will have to collect historical equities' price data. We will be concentrating on the South African market. This will be obtained from industry with the help of our supervisor, Rowan Douglas, or otherwise using the available library resources at our disposal, such as the Bloomberg terminal.

We plan to perform a back test on this past data to assess the efficacy of the diversification strategies based on the alternative risk measures we shall be investigating. This will be compared against the benchmark of the diversification strategy based on traditional MPT. If any similar investigations are found in the course of our research, further comparisons can be made.  The programming language used will be R.
	

\section{Paper Structure}
\label{sec:struc}

\be
\item Introduction
\item Background \& Theory
	\be
	\item Alternative Risk Measures
	\item Diversification Methods
	\ee
\item Methodology \& Data
\item Results
\item Discussion \& Conclusion
\item Appendices %can we swap Appendices and References?
\item References
\ee
 
\section{Division of Work}
\label{sec:dow} 

This is a group project. Collaborative work allows for opportunities of synergy following from a greater scope of knowledge and skills. It does, however, come with its own challenges. Richard Montgomery will be taking charge of the theoretical aspect of the project, such as the literature review. Tayla Radmore will be heading the practical, coding aspect of the project. Thomas K{\"o}nigkr{\"a}mer will act as an intermediary for these two aspects of the project: assisting in research, writing or coding, editing writing and workings, and ensuring cohesion in the project.

\section{Time-line}
\label{sec:time}

\bi
\item \textbf{Monday, 30 April 2018:} Any individual work for the literature review is to be completed by this date. Up until the next deadline, we will bring our individual parts together into a cohesive whole.
\item \textbf{Friday, 4 May 2018:} Soft deadline for literature review - to be sent to our supervisor. Over the weekend, each group member will individually inspect the draft.
\item \textbf{Monday, 7 May 2018:} Group editing of draft after individual editing over the weekend. If necessary, a meeting will be set up with our supervisor as soon as possible.
\item \textbf{Friday, 11 May 2018:} Submit literature review
\item \textbf{Pre-June Exams:} Utilise this time to:
	\bi
	\item Collect the necessary data
	\item Start work on the diversification methods to be used
	\item Possibly perform a back test, using the historical data, for the traditional MPT strategy. This result will be used as a benchmark against which the other strategies will be compared. 
	\ei
\item \textbf{June/July Vac:} Utilise this time to:
	\bi
	\item Perform the back tests for the alternative strategies
	\item Assess how these strategies performed in comparison to the MPT strategy.
	\ei
\item \textbf{Monday, 23 July 2018:} Any individual work for the draft is to be completed by this date. Up until the next deadline, we will bring our individual parts together into a cohesive whole.
\item \textbf{Friday, 27 July 2018:} Soft deadline for draft paper - to be sent to our supervisor. Over the weekend, each group member will individually inspect the draft.
\item \textbf{Monday, 30 July 2018:} Group editing of draft after individual editing over the weekend. If necessary, a meeting will be set up with our supervisor as soon as possible.
\item \textbf{Friday, 10 August 2018:} Submit draft paper 
\item \textbf{Friday, 7 September 2018:} Feedback for submitted draft. The next month will be used to improve our project, which will hopefully only require polishing.
\item \textbf{Friday, 19 October 2018:} Submit final paper 
\ei




\newpage
\bibliographystyle{natbib}
\bibliography{Bibliography}
\label{bib:bibliography}
 
\end{document}
