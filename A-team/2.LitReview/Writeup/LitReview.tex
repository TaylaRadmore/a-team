\documentclass[12pt,a4paper]{article}
\usepackage{natbib}
\usepackage[british]{babel}
\usepackage{parskip}
\usepackage{amsmath}
\usepackage{graphicx}
\usepackage{hyperref}
\usepackage{url}
\usepackage{amsfonts}
\usepackage{bbold}

%\usepackage{appendix}

%Any global settings go here
\graphicspath{{Figures/}}

\bibpunct[:]{(}{)}{;}{a}{,}{,}

\renewcommand{\baselinestretch}{1.5}

%some shortcut commands
\newcommand{\bi}{\begin{itemize}}
\newcommand{\ei}{\end{itemize}}
\newcommand{\be}{\begin{enumerate}}
\newcommand{\ee}{\end{enumerate}}
\newcommand{\e}[1]{{\mathbb E}\left[ #1 \right]}
\newcommand{\ts}{\textsuperscript}

\begin{document}

\begin{titlepage}

\title{Investigating the effectiveness of diversification strategies based on alternative risk measures\\
\line(1,0){250}\\
\large Literature Review}

\date{\today}

\author{Richard Montgomery (MNTRIC006)\\
Tayla Radmore (RDMTAY001)\\
Thomas K{\"o}nigkr{\"a}mer (KNGTHO005)}

\maketitle

\begin{figure}

\begin{center}
\includegraphics[width=3cm]{UCTlogo.jpg}\\[1cm]
\end{center}

\end{figure}

\begin{center}
\line(1,0){150}\\
{\tt mntric006@myuct.ac.za}\\
{\tt rdmtay001@myuct.ac.za}\\
{\tt thomasekng@gmail.com}
\end{center}

\thispagestyle{empty}

\end{titlepage}

\pagenumbering{roman}

\tableofcontents

\newpage

\pagenumbering{arabic}

\section{Introduction}
\label{sec:Intro}

Due to the critical financial events in our recent past, more and more attention is given to risk and loss mitigation \citep{righi2017simulation}. Diversification strategies are used to allocate assets to a portfolio in the most optimal way. The majority of literature on the matter focuses specifically on diversification strategies which use volatility (i.e. variance) as the risk measure. This is in spite of the fact that its use as a risk measure has come under considerable scrutiny. This has led to a need for research into the use of alternative risk measures (ARMs) in diversification strategies. Ultimately, we intend to answer the question: For a risk-averse investor holding long positions in South African stocks, can we diversify a portfolio using alternative risk measures that outperform the mean-variance (MV) portfolio?

With this literature review, we aim to create the necessary contextual base for our project. We begin by giving background information regarding diversification based on variance. This will be covered in section \ref{sec:Back}. Section \ref{sec:Risk} provides an evaluation of various risk measures that are available to us. Then, in section \ref{sec:Div}, we consider possible diversification techniques. Finally, section \ref{sec:Further} considers research methodologies by evaluating similar work that has already been done in this area.

\section{Background}
\label{sec:Back}

The MV diversification method seeks to minimise variance for a given expected return (or similarly, maximise expected return for a given variance) \citep{markowitz1952portfolio}. This is simple and seems to make financial sense. \cite{hoe2010empirical} describe to us the drawbacks of using this model. MV depends on the assumption that either:
\be
\item The asset returns are multivariate normally distributed; or
\item That investor has a quadratic utility function.
\ee
These are unrealistic assumptions which rarely occur in practice. This necessitates a more considered choice of risk measure to one that more accurately reflects reality. \cite{byrne2004different} criticise MV's use, since it would imply that the investor is indifferent between the risk that results in a return above the mean (or some benchmark), and the risk that results in a return below that same mean (or benchmark). This is an important and intuitive criticism against using MV diversification. 

\section{Risk Measures}
\label{sec:Risk}

In a paper with a similar goal to ours, \cite{byrne2004different} also sought to evaluate the use of ARMs. Due to the plethora of ARMs available today, they argue that this is a reason for why variance is still so widely used: investors simply avoid the difficult decision of choosing a risk measure. \cite{righi2017simulation} went further in their research as they evaluated 11 of the most commonly used risk measures. Their research, which uses historic data from US stocks, shows that there is no obviously dominant risk measure. However, they do concede that each risk measure is unique in its construction and interpretation. This idea is reflected by the ``best practices'' risk measures, briefly discussed by \cite{dowd2006after}. They described the work done by \cite{dhaene2003economic}, which argues that there is no universally ``best'' risk measure, since one needs to take the circumstances of the particular scenario into consideration.

Taking this information into account, we aim to choose a risk measure that satisfactorily suits a risk-averse investor. Despite the fact that risk-aversion theory is not a perfect or universally accepted idea, we believe that it is a good model. We begin our search by first evaluating and criticising some commonly used ARMs. This prompts us to seek less conventional, but arguably more suitable, measures.

\subsection{Value-at-Risk}
\label{subsec:VaR}

VaR is simply defined as the maximum loss  of a portfolio for a given confidence interval, and is widely used in industry \citep{consiglirisk}. \cite{dowd2006after} show that VaR has several positive attributes, such that it can be used to compare portfolios that are not restricted to a certain type of asset. Furthermore, it takes all risk factors that affect the portfolio into account, meaning that no simplifying assumptions are needed and that the factors do not have to be added incrementally to the risk measure to avoid any complications. Finally, \cite{dowd2006after} identify the probability linked units of measure of VaR as intuitive to understand.

Despite this, \cite{dowd2006after} believe that VaR is an imperfect measure. They justify replacing it with an ARM, because doing so would require very little extra work and would result in a more sound measure of risk. Their main criticism of VaR is that it gives no information beyond the worst case scenario that it describes. \cite{dowd2006after} go further in saying that this lack of information in the tail loss region reflects an unrealistic risk-seeking behaviour in the investor. \cite{ACERBI20021505} supports the view of \cite{dowd2006after}, by stating that VaR does not meet the axioms of coherence (discussed further in section \ref{subsec:ES}). We agree that VaR is a sub-optimal choice, as we seek a risk measure that better reflects risk-averse behaviour.

\subsection{Coherent Risk Measures - Expected Shortfall}
\label{subsec:ES}

\cite{dowd2006after} suggest an alternative to VaR in the form of coherent risk measures, of which Expected Shortfall (ES) is a special case \citep{ACERBI20021505}. The risk measure axioms of coherence were postulated by \cite{artzner1999coherent} in order to ensure that risk is more effectively managed. These axioms include: (1) Monotonicity; (2) Sub-additivity; (3) Positive homogeneity; and (4) Translation invariance.

\cite{CHEN20111777} criticise coherent risk measures because of the axiom of positive homogeneity, as it implies an irrational linear utility for the investor. It is due to this reason that \cite{Föllmer2002} introduced convex measures where they replace the properties of positive homogeneity and sub-additivity with convexity 

ES is defined as the expected loss beyond the maximum loss defined by VaR \citep{consiglirisk}. \cite{dowd2006after} describe it as an easily generated risk measure. They do, however, draw attention to the fact it suggests the investor is risk-neutral over the lower tail region it describes. This makes ES ill-suited for our choice of risk measure, since investors are generally attributed with risk-averse appetites for risk.

\subsection{Spectral Risk Measures - Conditional VaR}
\label{subsec:Spec}

\cite{dowd2006after} further discuss spectral risk measures, which combine the properties of coherence risk measures with risk-aversion theory. \cite{dowd2006after} have their own reservations regarding risk-aversion theory, since (as mentioned in section \ref{sec:Risk} above) it is neither perfect nor accepted universally. General spectral risk measures would require us to choose a risk-aversion function for the investor. This is a subjective decision that is difficult to make.

Conditional VaR is a special case of a spectral risk measure \citep{BRANDTNER20135526}. Its risk-averse nature would make it ideal for further research in this paper. However, \cite{BRANDTNER20135526} shows that spectral risk measures tend to corner solutions. Additionally, if there is a risk free asset there is no diversification and if there is not a risk free asset spectral measures provides limited diversification.

\subsection{Distortion Risk Measures}
\label{subsec:Distortion} 

Distortion measures are the last risk measures examined by \cite{dowd2006after}. They focus especially on a generalisation of the Wang Transform. They believe that it is a particularly useful measure due to its ability to recover the Capital Asset Pricing Model (CAPM) as well as the Black-Scholes model. They further describe it as a measure superior to both VaR and ES, since it takes the lower tail region of losses into account. It is uncertain whether this risk measure is ideal for our specific research.


\subsection{Weighted Expected Shortfall}
\label{subsec:WES}

\cite{CHEN20111777} propose an interesting ARM that avoids the drawbacks of coherence risk measures (described in section \ref{subsec:ES} above) and satisfies convexity and monotonicity.

\cite{CHEN20111777} believe the main concern for portfolio selection is that the investor's preferences are held consistently. Hence, they introduce Weighted Expected Shortfall (WES). WES is defined at a given confidence interval (say $\alpha$) as:
$$
WES_{\alpha}(X)= \alpha^{-1}(W(x_{\alpha})x_{\alpha}(\Pr[X \leq x_{\alpha}]-\alpha)-\e{W(X)X\mathbb{1}_{(X \leq x_{\alpha})}}),
$$ 
where 
$$ x_{\alpha} = \inf( {x \in R : \Pr[X \leq x] \geq \alpha })$$
Furthermore, $w(x)$ is:
\bi
\item A monotonically non-increasing function of x
\item Positive and convex for $x \leq 0 $
\item Non-negative and concave for $x>0 $
\ei 

$w(x)$ is a non-linear weight function. It attempts to reflect the asymmetry of financial asset returns that one would expect in reality \citep{CHEN20111777}. Thus, the choice of $w(x)$ depends on the investor's appetite for risk, and so perfectly suits what we wish to achieve with our choice of ARM. Various choices for $w(x)$ exist, but \cite{CHEN20111777} use:
\begin{gather}
w(x)= e^{-\lambda x}, \\
\text{	when	}  x \leq 0
\text{	and	} w(x) = 0 \text{	for	} x > 0. 
\end{gather}
They justify the use of this specific $w(x)$ by stating that '' [it] is highly plausible and easy to interpret". \cite{CHEN20111777} further state that taxes, and incomes can also be included in this model to make it more practical and useful. $\lambda$ must also be chosen. We can adjust its value to see its affect on the diversification and performance of the portfolios we create. A  higher $\lambda$ would correspond to a more risk averse investor\citep{CHEN20111777}.

\section{Diversification Strategies}
\label{sec:Div}

Risk measures only comprise a portion of our research. The difficult task will be finding a suitable diversification strategy to use given the ARMs we will be using. From the work done by \cite{bruder2012managing}, we have decided to focus on three methods of portfolio diversification.

\subsection{The Equally-Weighted Portfolio}
\label{subsec:EW} 

The method of Equally-Weighted (EW) portfolios is mentioned in the paper by \cite{bruder2012managing}. It is primarily used as a point of comparison for other alternative diversification methods. The most notable being the Risk Budgeting (RB) approach. It is a na{\"i}ve diversification method, allocating an equal weight to each asset comprising the portfolio. This is easily understood from its descriptive name. Due to this comparison in different diversification methods they performed, it is worth considering something similar for our research. We have continuously come across work which suggests that there is not an obviously superior risk measure (please see section \ref{sec:Risk} above). Perhaps the different diversification methods we can use is what should be investigated more closely once our ARMs have been chosen.

\subsection{Risk Minimisation}
\label{subsec:Riskmin} 

This is a simple and intuitive way of allocating asset weights in a portfolio. The MV diversification method, for example, does this by ensuring that its variance is minimised for a certain level of return (or vice versa) \citep{markowitz1952portfolio}. This can be simply applied to other risk measures as well. \cite{CHEN20111777} suggest this for their own risk measure, WES. They made the additional point that the environment surrounding the investment must be thoroughly investigated. In particular they set the required return at a reasonable level. This will mean that for the purposes of this paper the South African market will need to be investigated in order to be able to make realistic decisions going forward. \cite{righi2017simulation} diversified their portfolios (based on 11 different ARMs) by both minimising the risk, and maximising the ratio between the expected return and the risk.

\subsection{The Risk Budgeting Approach}
\label{subsec:RB} 

\cite{bruder2012managing} describe the RB approach as an ``alternative indexing strategy''. The most important part of their paper is the single sentence in which they state that this diversification strategy can be generalised so that it can be used for a number of different risk measures, provided the risk measure is both convex, and satisfies the Euler decomposition. \cite{bruder2012managing} did not explore this train of thought any further, but rather performed back-tests comparing the returns of indices which make use of different diversification techniques. Their tests used variance as the risk measure, and found that RB diversification diversifies portfolios more effectively than the MV optimisation approach. It found that the MV approach was heavily concentrated in a single asset class. They further show that the volatility of a RB portfolio is between the volatilities of the MV and EW portfolios respectively.

Let there be a portfolio consisting of $n$ assets, where $x_i$ represents the weighting of the $i^{th}$ asset ($i=1,2,3,...,n$). If a risk measure $\mathbb{P}$ is convex and satisfies Euler's Decomposition, then:
$$
\mathbb{P}=\sum_{i=1}^{n}x_{i}\frac{\partial\mathbb{P}}{\partial{x_{i}}}
$$
The $x_i$ factor represents the weight or exposure of the $i^{th}$ asset and the partial derivative that follows is the marginal risk of that same asset. We thus have the risk contribution of the $i^{th}$ asset ($RC_i$) as:
$$
RC_i=x_{i}\frac{\partial\mathbb{P}}{\partial{x_{i}}}
$$
Setting $RC_i=b_i$ gives us a set of n risk budgets, giving us our RB portfolio. \cite{bruder2012managing} made all $b_i$'s equal, thereby creating the Equal Risk Contribution (ERC) portfolio. This a useful simplification that we should adopt should we use the RB approach. \cite{bruder2012managing} show that:
$$
\mathbb{P}(x_{min risk}) \leq \mathbb{P}(x_{RB}) \leq \mathbb{P}(x_{EW}),
$$
where
$\mathbb{P}(x_{min risk}) \text{, } \mathbb{P}(x_{RB}) \text{, and } \mathbb{P}(x_{EW})
$
are the risks of the minimum risk, RB and EW portfolios respectively. 

\section{Further Considerations to Make}
\label{sec:Further}

Given our choice of ARMs and diversification methods, we must begin to consider the methodology moving forward. Here we look at what are common practices, constraints, and methods of measuring and comparing performance. 

\subsection{Share Selection \& Constraints}
\label{subsec:Constraints}

We shall be using historical, South African share data on which we shall be performing a back-test. We assume that the investor is risk-averse and therefore weights the risk of an expected return below some benchmark more heavily than an expected return above the same benchmark. This agrees with the opinion given by \cite{CHEN20111777}, that risk is asymmetric about the mean. 

A common constraint to set ourselves, as was done by \cite{righi2017simulation}, is to restrict ourselves to only holding long positions in the stock, and ensuring that all capital is allocated (i.e. that the sum of the weights equals 1). This simplifies the practical component of the project as well as the interpretation of the results. Additionally, short sales are generally infeasible for the normal investor, due to high costs and/or regulations \citep{CHEN20111777}.

We will need a method to select assets from our pool of data. \cite{righi2017simulation} performed their own research by randomly selecting 4, 16 and 64 stocks from the US equity market. This allowed them to see if risk measures allocated asset weightings differently when applied to varying amounts of assets. \cite{righi2017simulation} added another factor that may affect performance by varying the amount of trading days (125, 250 and 500) used to determine the diversification parameters. 

\cite{CHEN20111777} performed their research differently by randomly selecting a riskless asset together with 30 risky shares from the A-share stocks in the Chinese stock markets. They determined their diversification parameters using 600 trading days. They did not rebalance the portfolio, but merely observed its value after 1 and 4 weeks. 


\cite{righi2017simulation} found that re-balancing their portfolio at regular intervals did not result in significant changes. \cite{CHEN20111777} believe that a buy-and-hold strategy may be suitable for an emerging market, as investors may not need to constantly adjust their investment strategies. South Africa is seen more as an emerging market so a buy-and-hold strategy may well be sufficient. \cite{KRATZ2018393} provides more guidance on the matter of re-balancing. In their research, they re-balanced every 10 days. This corresponds to every two trading weeks. They described this is a reflection of normal practice. Interestingly, \cite{KRATZ2018393} included an ``oracle'' trader which they defined as a ``forecaster [that] knows the correct model and its exact parameter values''.  Re-balancing and an ``oracle'' trader are considerations we must make for our research. The ``oracle'' in particular would provide a comparison for the outcomes of our models to that of a portfolio with perfect prediction.

Often, transaction costs and taxes are excluded from these models for simplification reasons. It may also complicate the interpretation of the results. \cite{CHEN20111777} maintains though, that these are important considerations that investors take into account when choosing their portfolio. They go further to explain how costs and taxes can be included in models in a simplified way. This is a consideration we must make for our own research. Furthermore \cite{CHEN20111777} impose another constraint, namely ``upper and lower bounds on the proportion of the wealth that the investor will invest on a certain asset". They believe this is a worthwhile constraint because of institutional
restrictions that would dictate this in practice. This is likely too advanced a consideration for this particular paper and instead be something for consideration in further studies.

\subsection{Measuring \& Comparing Performance}
\label{subsec:Performance}

Measuring performance is integral to our paper as it is what we shall use to determine whether our ARMs and diversifications were effective. 

Performance can be simply measured by return and then compared, as was done by \cite{bruder2012managing} in their comparison between a capital-weighted index, and one that was weighted according to ERC.

\cite{CHEN20111777} use ratios to assess the portfolios, of which the most intuitive is the  return to $WES_{\alpha}$ ratio. This is is a ratio of mean return to risk (giving mean return per unit of risk) which was also used by \cite{hoe2010empirical}. 

\cite{byrne2004different} makes a good argument to look at the portfolio compositions rather thatn the risk-return trade-off.  \cite{righi2017simulation} did not look at the composition of the portfolios they created, and did not find any risk measure which was superior to the other. \cite{byrne2004different} found, however, that different ARMs give very different portfolio compositions, and this must be interpreted as differences in diversification. \cite{byrne2004different} argue that thus the choice of risk measure must reflect the investor's appetite for risk. \cite{CHEN20111777} use the Herfindahl index and the number of stocks included in the portfolio to measure the extent of diversification. The Herfindahl index is smaller the more diversified the portfolio is. This is a useful index which should be investigated further for our project. 

Finally, we must determine whether there is any significant change in performance. A hypothesis test is a conventional and easy way to determine this, as was done by \cite{righi2017simulation} in their comparable research.

\section{Conclusion}
\label{sec:Concl}


\newpage
\bibliographystyle{natbib}
\bibliography{Bibliography}
\label{bib:bibliography}
 
\end{document}
