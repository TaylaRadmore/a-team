\documentclass[12pt,a4paper]{article}
\usepackage{natbib}
\usepackage[british]{babel}
\usepackage{parskip}
\usepackage{amsmath}
\usepackage{graphicx}
\usepackage{hyperref}
\usepackage{url}
%\usepackage{appendix}

%Any global settings go here
\graphicspath{{Figures/}}

\bibpunct[:]{(}{)}{;}{a}{,}{,}

\renewcommand{\baselinestretch}{1.5}

%some shortcut commands
\newcommand{\bi}{\begin{itemize}}
\newcommand{\ei}{\end{itemize}}
\newcommand{\be}{\begin{enumerate}}
\newcommand{\ee}{\end{enumerate}}

\begin{document}

\begin{titlepage}

\title{Investigating the effectiveness of diversification strategies based on alternative risk measures\\
\line(1,0){250}\\
\large Proposal}

\date{\today}

\author{Richard Montgomery (MNTRIC006)\\
Tayla Radmore (RDMTAY001)\\
Thomas K{\"o}nigkr{\"a}mer (KNGTHO005)}

\maketitle

\begin{figure}

\begin{center}
\includegraphics[width=3cm]{UCTlogo.jpg}\\[1cm]
\end{center}

\end{figure}

\begin{center}
\line(1,0){150}\\
{\tt mntric006@myuct.ac.za}\\
{\tt rdmtay001@myuct.ac.za}\\
{\tt thomasekng@gmail.com}
\end{center}

\thispagestyle{empty}

\end{titlepage}

\pagenumbering{roman}

\tableofcontents

\newpage

\pagenumbering{arabic}

\section{Introduction}
\label{sec:Intro}

Asset allocation is an important aspect of creating a portfolio of assets. This process is commonly referred to as the diversification strategy. The majority of literature on the matter, however, focuses specifically on diversification strategies which use volatility as the risk measure. With this project, we intend to assess the efficacy of diversification strategies based on alternative risk measures.
This research is important, because the use of volatility as a risk measure has come under scrutiny of late (add reference here).
% sorry Rich - I started on this a bit because I was unsure if some of the things that I was including in the method should actually be mentioned in the intro, so I just popped in here and added them.




\section{Research Question}
\label{sec:ResQues}


\section{Literature Review}
\label{sec:LitRev}

\subsection{Core Literature}
\label{subsec:CorLit}

To date some quantile-based risk measures have been looked into. Namely "Value-at-Risk (VaR), coherent risk measures, spectral risk measures, and distortion risk measures" \citep{dowd2006after}.

\cite{dowd2006after} believe that VaR is an imperfect measure. It may well still be reasonable to look into VaR as it is a well known measure and is not particularly complicated for someone in the relevant field of work to understand. Looking further into VaR, \cite{dowd2006after} feel that there are some positive aspects. Namely that VaR can be used to compare portfolios that are not restricted to a certain type of asset; VaR takes into account the portfolios internal correlations in turn allowing a consideration of number of variables changing at once and lastly it's result is easy to understand and has a linked probability.  \cite{dowd2006after} believe a significant limitation of VaR is that it gives no information about what happens if that worst case lower tail end occurs and they feel that this makes this particular measure a poor measure to be used as a risk target measure. \cite{ACERBI20021505} believe VaR is not a good measure because it does not meet the axioms of coherence.

The next group of risk measures looked into were coherent risk measures and in particular Expected Shortfall \citep{ACERBI20021505}. \cite{dowd2006after} feel that this is still a measure that is easy to generate. They do, however, draw attention to the fact that this particular measure suggests the investor is risk-neutral past that lower tail, this may be seen as problematic. Expected Shortfall will ideally be looked into in much greater detain before starting back-testing.

This progressed to looking into spectral risk measures as these measures align with coherence and risk aversion theory \citep{dowd2006after}. Here \cite{dowd2006after} highlight a number of issues with risk aversion theory, which will not be expanded on at this point but it is worthwhile to keep in mind that risk aversion theory is not a perfect or universally accepted idea. Ideally spectral risk measures will be looked into in much greater detail going forward.

Lastly Distortion measures were briefly looked into, in particular the Wang Transform and a generalisation of it  \citep{dowd2006after}. \cite{dowd2006after} believed that a strength of this measure is its ability to recover the Capital Asset Pricing Model as well as Black-Scholes and that it is a superior measure compared to expected shortfall.
 
It is worth mentioning that \cite{dowd2006after} state in their paper that with regards to risk measures it may often be a case of the best measure for a particular application and not a case of there being a certain risk measure that is better than all the others. 

A justification for replacing VaR with one of these other measures provided by \cite{dowd2006after} is that it would require very little extra work. 

\subsection{Further Reading}
\label{subsec:FurRead}

In addition to the works discussed, further ideas should be explored. Firstly, this includes delving into the aforementioned risk measures more deeply. Perhaps the use of other risk measures should be considered as well. Secondly, diversification methods to be used will have to be investigated more comprehensively. This, again, will either entail expanding on our understanding of RB diversification methods, or either discovering other viable methods. 

\cite{dowd2006after} mentioned a number of other risk measures. Thus, they have provided us with a list of other possible risk measures which may be worthwhile to study more deeply:

\bi
\item Convex risk measures
\item Dynamic risk measures
\item Comonotonicity approaches
\item Markov bounds approaches
\item “Best practices” risk measures
\ei
 
``Best practice'' risk measures (which were also mentioned in section \ref{subsec:CorLit} above) may be particularly interesting to look into during the literature review of this paper. These risk measures focus on the idea that the context of the problem may be the most important factor to consider \citep{dowd2006after}. This may, however, be beyond the scope our research question.

A particular paper that is worth drawing attention to is  \textit{Conditional Value-at-Risk, spectral risk measures and (non-)diversification in portfolio selection problems - A comparison with mean–variance analysis}, by Mario Brandtner (2013). This paper is useful in that it seems to have a similar approach to what we have in mind.

Further useful readings which we may to consult include: 
\bi
\item  \textit{The Properties of Equally Weighted Risk Contributions Portfolios}, by Maillard S., Roncalli T. and Teiletche J. 
\item \textit{Toward Maximum Diversification}, by Choueifaty Y. and Coignard Y.
\item \textit{Risk Budgeting: A New Approach to Investing}, by Rahl L.
\item \textit{Active Portfolio Management: A Quantitative Approach for Producing Superior Returns and Controlling Risk}, by Grinold R. and Kahn R.
\item \textit{Risk and Asset Allocation}, by Meucci A.
\item \textit{Challenges and standards in integrating surveys of structural variation.}, by Scherer S.W. et al.
\item \textit{The Markowitz Optimization Enigma: Is ‘Optimized’ Optimal?}, by Michaud R.O.
\item \textit{Coherent Measures of Risk}, by Artzner P. et al.
\item \textit{Coherent Representations of Subjective Risk Aversion}, by Acerbi C. 
\item  \textit{On Law Invariant Coherent Risk Measures}, by Kusuoka S.
\item  \textit{Optimal Rules for Ordering Uncertain Prospects}, by Bawa V.S.
\item \textit{Mean-Risk Analysis With Risk Associated With Below-Target Returns}, by Fishburn P.C.
\item  \textit{A Class of Distortion Operators for Pricing Financial and Insurance Risks}, by Wang S.S.
\item  \textit{Assessing VaR Accuracy}, by Dowd K.
\item  \textit{Measuring Market Risk}, by Dowd K. 
\item \textit{Bayesian Value at Risk: From Linear to Non-Linear Portfolios}, by Siu T.K. et al.
\ei

\section{Methodology}
\label{sec:Method}

First and foremost, we must devise diversification strategies using various alternative risk measures. We intend to focus on either three or four risk measures. Despite its flaws (discussed in section \ref{subsec:CorLit} above) VaR has been identified as an important risk measure to consider, due to its prominent use in industry today \citep{consiglirisk}. Since ES uses similar workings to VaR, it may be easy to include this risk measure in our research as well. Additionally, we would like to look more deeply into at least one other risk measure which seems unusual, yet useful and efficient, to us.

Given our current level of research, the RB diversification method seems to be the most feasible diversification method at our disposal. Of course, this means that we will have to ensure that the risk measures we use are compatible with this diversification technique (i.e. that they are convex and satisfy the Euler decomposition \citep{bruder2012managing}). Further research and readings will help in the specifics of using this approach, and may also bring other alternative diversification techniques to light.

We plan to perform a back test on the collected past data to assess the efficacy of the diversification strategies based on the alternative risk measures we shall be investigating. This will be compared against a benchmark: namely the diversification strategy based on volatility as its risk measure. If any similar investigations are found in the course of our research, further comparisons can be made.  The programming language used will be R.

\section{Paper Structure}
\label{sec:struc}

\be
\item Introduction
\item Background \& Theory
	\be
	\item Alternative Risk Measures
	\item Diversification Methods
	\ee
\item Methodology \& Data
\item Results
\item Discussion \& Conclusion
\item Appendices %can we swap Appendices and References?
\item References
\ee
 
\section{Division of Work}
\label{sec:dow} 

This is a group project. Collaborative work allows for opportunities of synergy following from a greater scope of knowledge and skills. It does, however, come with its own challenges. Richard Montgomery will be taking charge of the theoretical aspect of the project, such as the literature review. Tayla Radmore will be heading the practical, coding aspect of the project. Thomas K{\"o}nigkr{\"a}mer will act as an intermediary for these two aspects of the project: assisting in research, writing or coding, editing writing and workings, and ensuring cohesion in the project.

\section{Time-line}
\label{sec:time}

\bi
\item \textbf{Monday, 30 April 2018:} Any individual work for the literature review is to be completed by this date. Up until the next deadline, we will bring our individual parts together into a cohesive whole.
\item \textbf{Friday, 4 May 2018:} Soft deadline for literature review - to be sent to our supervisor. Over the weekend, each group member will individually inspect the draft.
\item \textbf{Monday, 7 May 2018:} Group editing of draft after individual editing over the weekend. If necessary, a meeting will be set up with our supervisor as soon as possible.
\item \textbf{Friday, 11 May 2018:} Submit literature review
\item \textbf{Pre-June Exams:} Utilise this time to:
	\bi
	\item Collect the necessary stocks' data
	\item Start work on the diversification methods to be used
	\item Possibly perform a back test, using the historical data, for the diversification strategy using volatility as its benchmark. This result will be used as a benchmark against which the other strategies will be compared. 
	\ei
\item \textbf{June/July Vac:} Utilise this time to:
	\bi
	\item Perform the back tests for the alternative strategies
	\item Assess how these strategies performed in comparison to the MPT strategy.
	\ei
\item \textbf{Monday, 23 July 2018:} Any individual work for the draft is to be completed by this date. Up until the next deadline, we will bring our individual parts together into a cohesive whole.
\item \textbf{Friday, 27 July 2018:} Soft deadline for draft paper - to be sent to our supervisor. Over the weekend, each group member will individually inspect the draft.
\item \textbf{Monday, 30 July 2018:} Group editing of draft after individual editing over the weekend. If necessary, a meeting will be set up with our supervisor as soon as possible.
\item \textbf{Friday, 10 August 2018:} Submit draft paper 
\item \textbf{Friday, 7 September 2018:} Feedback for submitted draft. The next month will be used to improve our project, which will hopefully only require polishing.
\item \textbf{Friday, 19 October 2018:} Submit final paper 
\ei

\newpage
\bibliographystyle{natbib}
\bibliography{Bibliography}
\label{bib:bibliography}
 
\end{document}
